\documentclass[10pt,a4paper]{article}
%\usepackage[english,spanish]{babel}
\usepackage{indentfirst}
\usepackage{anysize} % Soporte para el comando \marginsize
%\marginsize{1.5cm}{1.5cm}{0.5cm}{1cm}
\marginsize{2,5cm}{1,8cm}{4cm}{1,7cm}
\usepackage[psamsfonts]{amssymb}
\usepackage{amssymb}
\usepackage{amsfonts}
\usepackage{amsmath}
\usepackage{amsthm}
\usepackage{stackrel}
\usepackage{graphicx}

\usepackage[spanish]{babel}
\selectlanguage{spanish}
\usepackage[utf8]{inputenc} 

\usepackage{multicol}
\renewcommand{\thepage}{}
\columnsep=7mm

%%%%%%%%%%%%%%%%%%%%%%%%%%%%%%%%%%%%%%%%
\newtheorem{definicion}{Definici\'on}[section]
\newtheorem{teorema}{Teorema}[section]
\newtheorem{prueba}{Prueba}[section]
\newtheorem{prueba*}{Prueba}[section]
\newtheorem{corolario}{Corolario}[section]
\newtheorem{observacion}{Observaci\'on}[section]
\newtheorem{lema}{Lema}[section]
\newtheorem{ejemplo}{Ejemplo}[section]
\newtheorem{solucion*}{Soluci\'on}[section]
\newtheorem{algoritmo}{Algoritmo}[section]
\newtheorem{proposicion}{Proposici\'on}[section]

\linespread{1.4} \sloppy

\newcommand{\R}{\mathbf{R}}
\newcommand{\N}{\mathbf{N}}
\newcommand{\C}{\mathbb{C}}
\newcommand{\Lr}{\mathcal{L}}
\newcommand{\fc}{\displaystyle\frac}
\newcommand{\ds}{\displaystyle}

\DeclareMathOperator{\Dom}{Dom}

%%%%%%%%%%%%%%%%%%%%%%%%%%%%%%%%%%%%%%%%

\renewcommand{\thefootnote}{\fnsymbol{footnote}}
\usepackage{url}
\usepackage{hyperref}
\begin{document}
\begin{center}
 {\Large \textbf{MODELAMIENTO DE LA ECONOMÍA DE UN PAÍS MEDIANTE EL MODELO DE LEONTIEF }}
\end{center}
\begin{center}
 Gustavo Lozano, Miller Silva, Victor Ponce, Kevin Solano, Nicks Lazaro\vskip5pt
{\it Facultad de Ciencias, Universidad Nacional de Ingenier\'{\i}a\\}
\end{center}
%\maketitle 
\vspace*{1cm}
\begin{abstract}
\noindent El modelo de Leontief tambien conocido por el modelo de IO(Input-Output) se elabora a partir de datos económicos observados en una región, que puede ir desde una nación a una región dentro de la misma. Concierne por regla general a la producción industrial agrupada en sectores. La actividad económica en la región se divide en un número de segmentos o de sectores productivos. Pueden ser industrias en sentido más general (automóviles) o más específico como (industria de neumáticos). Cada sector agrupa actividades que tienen diferentes ritmos de consumo y producción de bienes. Parte de la producción de un sector (Output) puede ir al consumo (Input) de otro distinto sector dentro de la región bajo estudio. Esta información se recolecta en forma de una tabla denominada: Tabla Input-Output o Tabla IO. Las tablas con sus interdependencias se suelen elaborar con datos procedentes de intervalos anuales. Los intercambios de bienes suelen ser indicados como ventas, compras o bienes físicos. Pero es habitual que las unidades de medida empleados en el modelo se realice en términos monetarios. 

\end{abstract}

\begin{quotation}
	{\small
		
		Keywords: \\ 
		Tabla Input-Output\\
		Demanda externa\\
		Demanda interna\\
		Sistemas cerrados\\
		Sistemas abiertos\\
	}
\end{quotation}


\pagebreak

\begin{multicols}{2}
\begin{center}
{\large \bf 1. INTRODUCCI\'ON}
\end{center}
 Con el fin de comprender y ser capaz de manipular la economía de un país o una región, uno tiene que llegar a un cierto modelo basado en los diversos sectores de esta economía. El modelo de Leontief es un intento en esta dirección. Basado en la suposición de que cada industria en la economía tiene dos tipos de exigencias: la demanda externa (de fuera del sistema) y la demanda interna (demanda coloca en una industria por otro en el mismo sistema), el modelo de Leontief representa la economía como un sistema de ecuaciones de lineales. El modelo de Leontief fue inventado en los años 30 por el profesor Wassily Leontief (imagen de arriba) que desarrolló un modelo económico de la economía de Estados Unidos mediante su división en 500 sectores económicos. El 18 de octubre de 1973, el profesor Leontief fue galardonado con el Premio Nobel de economía por su esfuerzo.
\\
\begin{center}
	\includegraphics[scale=0.7]{leontief}
\end{center}

\vspace*{0.5cm}

\begin{center}
{\large \bf 2. CONCEPTOS PREVIOS}
\end{center}
Hay dos tipos de modelados de la economía que estableció Leontief: el modelo cerrado y el modelo abierto.\\
\begin{itemize}
	\item \textbf{EL MODELO CERRADO DE LEONTIEF} : Considerar una economía que consiste en $n$ industrias (o sectores)   $S_{1}, ... , S_{n}$. Eso significa que cada industria consume algunos de los bienes producidos por las otras industrias, incluso a sí misma (por ejemplo, una planta generadora de energía utiliza parte de su propia energía para la producción). Decimos que tal economía esta cerrada si satisface sus propias necesidades. Es decir, no hay mercancías que salgan o entren en el sistema. Sea $m_{ij}$ el número de unidades producidas por la industria $S_{i}$ y necesarias para producir una unidad de la industria $S_{j}$. Si $p_{k}$ es el nivel de producción de la industria $S_{k}$, luego $m_{ij}p_{j}$ representa el número de unidades producidas por la industria $S_{i}$ y consumidas por la industria $S_{j}$. Entonces, el número total de unidades producidas por la industria $S_{i}$ viene dado por:
\[  p_{1}m_{i1} + p_{2}m_{i2} + ... + p_{n}m_{in} \]
Para tener una economía equilibrada, la producción total de cada industria debe ser igual a su consumo total. Esto nos dará el siguiente sistema lineal:

\begin{equation*}
	\begin{matrix}
	m_{11}p_{1} &+& m_{12}p_{2} &+& \ldots  &+ &m_{1n}p_{n} &=& p_{1}\\
	m_{21}p_{1} &+& m_{22}p_{2} &+&\ldots  &+& m_{2n}p_{n} &=& p_{2}\\
	\vdots&&\vdots&&&&\vdots&&\vdots\\         
	m_{n1}p_{1} &+& m_{n2}p_{2} &+& ... & +& m_{nn}p_{n} &=& p_{n}
	\end{matrix}
\end{equation*}\\
Luego, nuestro sistema lineal será:       
 
 \[ A =
 \left( \begin{array}{cccc}
 m_{11} & m_{12} & \cdots & m_{1n} \\ 
 m_{21} & m_{22} & \cdots & m_{2n} \\
 \vdots & \vdots & \ddots & \vdots \\
 m_{n1} & m_{n2} & \cdots & m_{nn}
 \end{array} \right) \]\\
 
Entonces nuestro sistema a resolver se puede escribir como $AP = P$, donde:
	
 \[ P =
\left( \begin{array}{cccc}
p_{1} \\ 
p_{2} \\ 
\vdots  \\
p_{n} \\ 
\end{array} \right) \]\\
 $A$ es denominada MATRIZ DE ENTRADA-SALIDA.\\
 Luego estamos buscando un vector P que satisfaga $AP = P$ y con componentes no negativos, al menos uno de los cuales sea positivo.\\
 
 \item{\textbf{EL MODELO ABIERTO DE LEONTIEF: }}El primer modelo de Leontief trata el caso en el que no hay mercancías que ingresen a la economia, pero en realidad esto no sucede muy a menudo. Por lo general, una determinada economía tiene que satisfacer una demanda externa, por ejemplo, de organismos como los organismos gubernamentales. En este caso, sea $d_{i}$ la demanda de la industria exterior $Si$, $p_{i}$ y $m_{ij}$ se definen como en el modelo cerrado. Luego tendremos lo siguiente:
 $$ p_{i} = m_{i1}p_{1} + m_{i2}p_{2} + ... + m_{in}p_{n} + d_{i}$$
 para cada $i=1,2,...,n$. Esto nos da el siguiente sistema lineal:  $P = AP + d$, donde $P$ y $A$ son definidos como en el modelo cerrado y $d$ es el vector de demanda:\\
 	\[ d =
 	\left( \begin{array}{cccc}
 	d_{1} \\
 	d_{2} \\ 
 	\vdots  \\
 	d_{n} \\ 
 	\end{array} \right) \]\\
 	
 La manera de obtener nuestro sistema lineal es:
 	$$ P = AP +d \Rightarrow P-AP = d \Rightarrow (I-A)P = d$$
 	el cual sera el sistema a resolver.
 
 Existen varios métodos para resolver un sistema de ecuaciones, pero entre los más conocidos tenemos:
 \begin{itemize}
 	\item \textbf{Método de Gauss:} Consiste en hacer operaciones elementales a la matriz aumentada hasta llevarlo a una triangular superior
 	\begin{equation*}
 		\begin{bmatrix}
	 		a_{11}&a_{12}&a_{13}&\cdots&a_{1n}&|&b_{1}\\
	 		0&a_{22}&a_{23}&\cdots&a_{2n}&|&b_{2}\\
	 		0&0&a_{33}&\cdots&a_{3n}&|&b_{3}\\
	 		\vdots&\vdots&\vdots&\ddots&\vdots&|&\vdots\\
	 		0&0&0&\cdots&a_{nn}&|&b_n
 		\end{bmatrix}
 	\end{equation*}
 	la cual puede ser resuelta hallando los valores de las variables de abajo hacia arriba.
 	\item \textbf{Método Gauss-Jordan:} Dado el sistema $Ax=b$, lo que este método principalmente hace es hallar la inversa de $A$ para luego obtener la solución de la siguiente manera $$x=A^{-1}b$$ 
 	\end{itemize}
\end{itemize}
\newpage
%\vspace*{1cm}
\begin{center}
{\large \bf 3. AN\'ALISIS}
\end{center}
Vamos a trabajar con el modelo abierto ya que este se asemeja más a la realidad. Supongamos que tenemos los datos del comportamiento de un sistema abierto, de 20 industrias (en dólares), en la matriz $A $ y $d$: (click en el siguiente cuadro)

\href{https://nbviewer.jupyter.org/github/MillerSilva/numerical-analysis/blob/master/genera_sistema.ipynb}{Matriz A y d}
\\
\noindent Donde $a_{ij}$ es el dinero con el que $S_j$ compra productos a la industria $S_i$ para producir \$1 de producto y $d_i$ es la demanda exterior que tiene la industria $S_i$.

\noindent Ahora aplicando el modelo abierto de Leontief, nuestro sistema a resolver es $$(I-A)P=d$$ donde $P_i$ es la producción total(en dólares) de la industria $S_i$.
%\begin{center}
%{\large \bf 4. OBSERVACIONES}
%\end{center}
%
%\begin{center}
%{\large \bf 5. CONCLUSIONES}
%\end{center}

%\begin{center}
%{\large \bf Agradecimientos}
%\end{center}
%Los autores agradecen a las autoridades de la Facultad de Ciencias de la Universidad Nacional de 
%Ingenier\'{\i}a por su apoyo.
%%\begin{center}
%%{\large \bf Apendice: }
%%\end{center}

\end{multicols}

%\begin{center}
% -----------------------------------------------------------------------------
%\end{center}
\begin{multicols}{2}
%\begin{list}{}{\setlength{\topsep}{0mm}\setlength{\itemsep}{0mm}%
%\setlength{\parsep}{0mm}\setlength{\leftmargin}{4mm}}
%%
%%------------------------------------- References --------------------
%\small
%\item[1.] I.K. Argyros, \textit{Newton-like methods under mild \linebreak differentiability conditions with error analysis,} Bull. \linebreak Austral. Math. Soc. \textbf{37} (1988), 131-147.
%\item[2.] 
%%---------------------------------------------------------------------
%%
%\end{list}
\end{multicols}
\end{document}%\grid

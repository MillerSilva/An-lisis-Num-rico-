\documentclass[10pt,a4paper]{article}
\usepackage[utf8]{inputenc}
\usepackage[spanish]{babel}
\usepackage{amsmath}
\usepackage{amsfonts}
\usepackage{amssymb}
\usepackage{graphicx}
\usepackage[left=2cm,right=2cm,top=2cm,bottom=2cm]{geometry}
%%%%%%
\newcommand{\ds}{\displaystyle}
\newcommand{\re}{\mathbb{R}}

\newtheorem{theo}{Teorema}[section]
\newtheorem{example}{Ejemplo}[section]
\newtheorem{definition}{Definición}[section]
\newtheorem{algorithm}{Algoritmo}[section]
%%%%%%%
\title{Series de Neumann}
\begin{document}
\maketitle

\section{Introducción}

Sea $V$ un espacio vectorial, al cual se le asigna una norma $\Vert .\Vert$, decimos que $(V, \Vert .\Vert)$ es un espacio lineal normado.
La noción de convergencia de una sucesión de vectores $v^{(1)}, v^{(2)}, v^{(3)},\ldots$ se define de la siguiente manera:
$$[v^{(k)}]\rightarrow v\quad sii\quad\lim_{k\rightarrow\infty}\Vert v^{(k)}-v\Vert = 0$$

\section{Series de Neumann}

\begin{theo}
	Si $A$ es una matriz $n\times n$ tal que $\Vert A\Vert <1$, entonces $A^{-1}$ es inversible y
	\begin{equation}\label{neuman's serie}
		(I-A)^{-1} = \sum_{k=0}^{\infty}A^{k} 
	\end{equation}
\end{theo}

\textbf{Prueba}\\
Sea 
\begin{align*}
	S_{m} 	= (I-A)\ds\sum_{k=0}^{m}A^{k} &= \ds\sum_{k=0}^{m}A^{k}-A^{k+1}\\
			&= I -A^{m+1}\\							 
\end{align*}

Vamos a probar que $(S_{m})$  converge a $I$
$$\lim_{m\rightarrow\infty}\Vert S_{m}-I\Vert = \lim_{m\rightarrow\infty}\Vert -A^{m+1}\Vert = \lim_{m\rightarrow\infty}\Vert A^{m+1}\Vert$$
Pero $\Vert A^{m+1}\Vert \leq\Vert A\Vert^{m+1}$
$$\Rightarrow\lim_{m\rightarrow\infty}\Vert S_{m}-I\Vert\leq\lim_{m\rightarrow\infty}\Vert A\Vert^{m+1} = 0\quad , pues\; \Vert A\Vert <1$$
\begin{equation}\label{convergencia_neumann}
\therefore (S_{m}) \rightarrow I
\end{equation}

Ahora probemos que $I-A$ es inversible.
Supongamos que $I-A$ no es innversible\\
$\Rightarrow\exists x\in \re^{n}-\{0\}/\quad (I-A)x = 0$\\
Ahora tomando $z = \frac{x}{\Vert x \Vert}$
\begin{align*}
	&\Rightarrow (I-A)z\Vert x\Vert = 0\\
	&\Rightarrow (I-A)z = 0,\quad\Vert z\Vert = 1\\
	&\Rightarrow 1 = \Vert z\Vert = \Vert Az\Vert\leq\vert A\Vert\Vert z\Vert = \Vert A\Vert\\
	&\Rightarrow 1\leq \Vert A\Vert\quad(\textbf{contradicción})
\end{align*}
$$\therefore I-A\quad es\; inversible$$

De $(\ref{convergencia_neumann})$ tenemos $(S_{m}) = \left((I-A)\ds\sum_{k=0}^{n}A^{k}\right)\rightarrow I$\\
Por lo tanto $\left(\ds\sum_{k=0}^{m}A^{k}\rightarrow (I-A)^{-1}\right)$, es decir
$$(I-A)^{-1} = \sum_{k=0}^{\infty}A^{k}$$

\begin{algorithm}
	VER COMO HACER LOS SANGRADOS
\end{algorithm}

\begin{example}[Ejercicio de Aplicación]
Use la se serie  de Neumann (\ref{neuman's serie}) para calcular la inversa de la matriz.
$$B = \begin{bmatrix}
	0.9	&	-0.2		&	-0.3\\
	0.1	&	1.0		&	-0.1\\
	0.3	&	0.2		&	1.1
\end{bmatrix}$$
Use la norma del máximo
\end{example}
\textbf{Ver solución en cuaderno Jupyter}

\begin{theo}
	Si $A$ y $B$ son matrices $n\times n$ tal que $\Vert I-AB\Vert <1$, $A$ y $B$ son inversibles. Entonces:
	\begin{align*}
		A^{-1} &= B\left(\sum_{k=0}^{\infty} (I-AB)^{k}\right)\\
		B^{-1} &= \left(\sum_{k=0}^{\infty} (I-AB)^{k}\right)A
	\end{align*}
\end{theo}
\textbf{Prueba}\\
Como $\Vert I-AB\Vert < 1$, entonces $AB = I-(I-AB)$ es inversible y
$$(AB)^{-1} = \sum_{k=0}^{\infty}(I-AB)^{-1}$$
Luego
\begin{align*}
	A^{-1} &= BB^{-1}A^{-1} = B(AB)^{-1} = B\left(\sum_{k=0}^{\infty} (I-AB)^{k}\right)\\
	B^{-1} &= BA^{-1}A = (AB)^{-1}A = \left(\sum_{k=0}^{\infty} (I-AB)^{k}\right)A
\end{align*}
\end{document}

